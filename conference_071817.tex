\documentclass[conference]{IEEEtran}
\IEEEoverridecommandlockouts
% The preceding line is only needed to identify funding in the first footnote. If that is unneeded, please comment it out.
\usepackage{cite}
\usepackage[utf8]{inputenc}
\usepackage{amsmath,amssymb,amsfonts}
\usepackage{algorithmic}
\usepackage{graphicx}
\usepackage{textcomp}
\def\BibTeX{{\rm B\kern-.05em{\sc i\kern-.025em b}\kern-.08em
    T\kern-.1667em\lower.7ex\hbox{E}\kern-.125emX}}
\begin{document}

\title{El Rol del Reconocimiento Facial en la Seguridad Empresarial\\
{\footnotesize }
}
\maketitle



\section{Resumen}
El uso de reconocimiento con rasgos corporales no es algo nuevo, desde tiempos de los egipcios se estampaban las palmas de la mano o la planta del pie para guardar registros y así identificar y garantizar autoría. Ahora en el siglo XX, surge la necesidad de tomar provecho de este tipo de características únicas e irrepetibles que todos tenemos.
Dado esta facilidad de seguridad tecnológica, se pretende acercar al lector al rol que juega el reconocimiento facial en la seguridad empresarial, tanto para controles de acceso como garantía de identificación irrefutable de individuos por medio de biometría. 
\begin{IEEEkeywords}
Biometría, rasgos, seguridad, reconocimiento facial
\end{IEEEkeywords}

\subsection{Abstract}
The use of body parts recognition has history from egyptians using hands palm as signature, also arm plant as secure method to save registers with his property. Now in 20th century, arise some needs in order to take advantage of this personal and unique characteristics of the human being. Given this facility of technology security, the intention is to bring the reader closer to the role of facial recognition in enterprise security, both for access controls and for accurate identification of every person by biometric ways.

\begin{IEEEkeywords}
Biometric, traits, security, face recognition
\end{IEEEkeywords}

\section{Definición de términos o acrónimos}
\begin{itemize}
\item Badge: Gafete electrónico 
\item Smartphone: teléfono inteligente
\item Hacking: Técnica de vulnerar sistemas o equipos
\end{itemize}

\section{Desarrollo}
\subsection{Control de Acceso}\label{AA}
Existen distintas maneras de controlar el acceso a lugares o sistemas a los cuales existen ciertas restricciones. Inicialmente se utilizaban las contraseñas para controlar estos accesos, luego se han comenzado a utilizar tokens, y en los últimos años se han venido implementando controles biométricos, los cuales, utilizados adecuadamente, brindan mayor seguridad y confiabilidad en el acceso, ya que existen ciertas características, como el iris del ojo, o las huellas dactilares, las cuales son únicas de cada ser humano, y brindan un mucho menor riesgo de falsificación de la cuenta. “Un sistema biométrico se basa en medir, comparar, codificar, transmitir, almacenar y/o reconocer un individuo por medio de sus características únicas, dando un resultado altamente confiable.” \cite{b1}.
En el ámbito empresarial a como avanzan los días, es necesario incursionar en la mejora constante, como es sabido las empresas que mejor se adapten a los cambios son las que tienen más probabilidades de tener éxito y estar adelante contra sus competidores. Debido a esto, la evolución en el reconocimiento facial como compañero del control de acceso nos facilita la labor, tanto para evitar los antiguos sistemas de uso de badge, como también garantizar la disminución de suplantación de identidad.


\subsection{Soluciones disponibles}
Entre las bondades que ofrece la tecnología de reconocimiento facial, se puede obtener datos para solventar 3 grandes necesidades; monitoreo de grupos de personas, análisis demográfico y autenticación en sistemas. 
En el mercado existen varias compañías que ofrecen soluciones de vanguardia para ayudar al sector empresarial a tener un mayor control tanto en control de acceso como análisis de usuarios. Podemos mencionar, algunas empresas de renombre que se dedican a este trabajo: 

\begin{itemize}
\item Innovatrics 
\item Axis communications
\item Casmar
\item Herta Security
\end{itemize}

\subsection{Implementaciones}
Podemos incluir, dentro de las implementaciones del reconocimiento facial, las aplicaciones o sistemas que necesitan de un alto control de seguridad, y a la vez que no le brinden complicaciones a los clientes. La mayoría de sistemas deberían de contar con una amplia confiabilidad y a la vez mucha facilidad y rapidez para su acceso, pero debido a que esto tiene un alto costo de implementación actualmente, los bancos y aplicaciones de seguridad son quienes tienen mayor interés hacia estas tecnologías. Dentro de las aplicaciones de seguridad, podemos encontrar cuerpos y fuerzas de seguridad del estado, gobiernos, aeropuertos, eventos, FBI, estadios deportivos, ciudades seguras.
Las implementaciones pueden variar dependiendo de la necesidad del cliente, si se utiliza un smartphone, la implementación va a depender del software propiamente y la empresa desarrolladora, mientras que, si la implementación se realiza en un restaurante o casino, ya entran en juego dispositivos como cámaras, discos duros, servidores y sensores que la empresa contratada ponga a disposición.


\subsection{Estado actual del reconocimiento facial en Costa Rica}

MasterCard utiliza la tecnología Identity Check, mediante la cual se implementa tanto el reconocimiento de huellas dactilares como también faciales \cite{b3}. El reconocimiento dactilar ya se encuentra adoptado en la aplicación de banca móvil del BAC, aprovechando las características del iPhone, pero sin embargo el reconocimiento facial aún no se utiliza. Debido a que la última versión de iPhone implementa reconocimiento facial, es probable que próximamente también el BAC aproveche esta capacidad.

La empresa Coopeservidores si ha apostado al reconocimiento facial, y lo utilizan tanto desde apps, quioscos y página web actualmente \cite{b3}.

El banco Nacional, y Davivienda también tienen iniciativas para utilizar reconocimiento facial, principalmente basados en la tecnología de www.facephi.com, para la cual también están basados Coopeservidores \cite{b4}.

\subsection{Riesgos}
Siempre va a existir el riesgo del hacking, por lo que después de un reconocimiento adecuado, se debe de validar por un ser humano. La tecnología de reconocimiento facial, se encuentra en una etapa de madurez en la cual han superado simples errores como cuando Android implementó por primera vez el reconocimiento de caras, y era fácilmente hackeado por fotos que se ponían en frente de la cámara. Actualmente Apple brinda esta tecnología con una mayor confianza, ya que dentro de las verificaciones incluyen análisis de profundidad de la foto, pero eso sí, todavía con retos a contrarrestar, como por ejemplo intentos de engaños con máscaras y texturas especiales.
Como todas las soluciones de seguridad, siempre existe cierto riesgo de error, por lo que, en la etapa actual del reconocimiento facial, se debe de contar con la verificación final de humanos para asegurarse que los resultados sean correctos. 


\subsection{Sugerencias}
Analizando diferentes aristas en el tema de la utilización de tecnología de reconocimiento facial, Helcio comparte en su publicación lo siguiente:  “el Departamento de Migración y Ciudadanía de Australia cuenta con un Programa Biométrico Nacional para el Control Fronterizo, el cual captura, reconoce y almacena imágenes faciales y huellas dactilares para los procesos de inmigración, ciudadanía y cruce fronterizo.” \cite{b2}
Es un ejemplo digno de imitar en las diferentes entidades migratorias de cada país, para robustecer la seguridad y a la vez ir eliminando procesos manuales que demandan gasto de recursos tanto humanos como materiales.
Por último, se debe tener muy en claro la legislación de cada país, para no pecar de ignorantes, y en vez de utilizar este tipo de tecnología biométrica para nuestro beneficio o el beneficio empresarial, nos veamos involucrados en alguna demanda o simplemente no podamos utilizar material recabado por un descuido previo.


\subsection{Conclusiones}
La tecnología de reconocimiento facial, algo visto como futurista desde hace muchos años, es toda una realidad hoy en día, y saca provecho de la capacidad de procesamiento de las 
computadoras para brindar diversas soluciones a la población. Implementando los avances de empresas líderes en el área, se pueden utilizar sus soluciones para 
facilitar tareas que manualmente implican muchas horas de trabajo, tal vez hasta sin solución. Como un ejemplo podemos tomar el buscar una foto de uno mismo en un álbum de fotos de facebook, por ejemplo en una carrera de atletismo que contiene miles de fotos. Esta tarea es tediosa y toma muchas horas, para la búsqueda de una sola persona. Soluciones como la de Herta Security brindan reconocimiento de individuos en tiempo real, por lo que una tarea que normalmente tomaría horas, con este software se realiza en segundos, brindando un gran ahorro de recursos y evitando incidentes graves dentro de empresas o eventos.
En resumen, las técnicas y aportes de la tecnología de reconocimiento facial aportan soluciones muy ricas en temas de control, reconocimiento y análisis en cuestión de segundos. Pero siempre su uso e implementación necesitará la revisión y cuidado en el tema de seguridad.


\begin{thebibliography}{00}
\bibitem{b1} Lopez Perez, N., and Toro Agudelo, J. (2012). Técnicas de biometría basadas en patrones faciales del ser humano. Proyecto de grado para optar al título de Ingeniero de Sistemas y Computación. Obtenido de http://repositorio.espe.edu.ec/handle/21000/7008
\bibitem{b2} Beninatto, H. (17 de 07 de 2014). Soluciones biométricas contra la inseguridad. Forbes.
\bibitem{b3} Chacón, K. (29 de Octubre de 2017). Banca da sus primeros pasos en el uso de la biometría para trámites. Obtenido de Sitio web de La Nación: https://www.nacion.com/economia/banca/banca-da-sus-primeros-pasos-en-el-uso-de-la-biometria-para-tramites/75GKPIUHDFDMLDOXN3SLLBVHYM/story/
\bibitem{b4} FacePhi. (s.f.). Casos de éxito. Obtenido de FacePhi - Beyond Bimetrics: https://www.facephi.com/es/content/costa-rica/.
\end{thebibliography}

\end{document}
